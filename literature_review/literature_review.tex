\documentclass[11pt]{article}

% --- Packages ---
\usepackage[utf8]{inputenc}
\usepackage[T1]{fontenc}
\usepackage{lmodern}
\usepackage[margin=2.5cm]{geometry}
\usepackage{amsmath,amssymb}
\usepackage{booktabs}
\usepackage{hyperref}
\usepackage{cleveref}
\usepackage{enumitem}
\usepackage{microtype}
\usepackage{longtable}
\usepackage{array}
\usepackage{tabularx}
\usepackage{natbib}
\bibliographystyle{unsrtnat}

\hypersetup{
    colorlinks=true,
    linkcolor=blue!70!black,
    citecolor=green!50!black,
    urlcolor=blue!70!black,
}

% Compact lists
\setlist{nosep,leftmargin=1.5em}

% --- Title ---
\title{Literature Review:\\Evolutionary Algorithms for Quantum Circuit Design}
\author{Thomas Snell}
\date{February 2026}

\begin{document}
\maketitle

\begin{abstract}
This literature review surveys the intersection of evolutionary algorithms and quantum circuit design as of early 2026.
We identify key trends, benchmark problems, active research groups, and open problems across eight sub-fields: EA-based circuit synthesis, quantum architecture search, QAOA parameter optimization, VQE ansatz design, multi-objective quantum optimization, quantum error correction, ML-based circuit design, and quality-diversity methods.
A gap analysis positions our project---a fast numpy-based comparative study of six optimization methods across six quantum problems---within the broader landscape, and identifies under-explored research directions including quality-diversity optimization, transfer learning, and co-evolutionary approaches.
\end{abstract}

\tableofcontents
\newpage

% ==============================================================================
\section{Field Overview}
\label{sec:overview}

The intersection of evolutionary algorithms (EAs) and quantum circuit design is experiencing a renaissance driven by the NISQ era's need for automated, hardware-aware circuit optimization. Three major trends dominate:

\begin{enumerate}
    \item \textbf{EAs remain competitive} with reinforcement learning and generative models for small-scale circuit synthesis, with advantages in interpretability and ease of implementation.
    \item \textbf{Multi-objective and noise-aware optimization} are becoming standard requirements.
    \item \textbf{Hybrid approaches dominate}: pure EA or pure RL approaches are being replaced by hybrids (EA + parameter optimization, EA + noise models, GA + QAOA).
\end{enumerate}

% ==============================================================================
\section{Key Recent Papers by Topic}
\label{sec:papers}

% --- 2.1 ---
\subsection{EA-Based Quantum Circuit Synthesis}

\begin{longtable}{p{5.5cm} p{2.5cm} p{1cm} p{5cm}}
\toprule
\textbf{Paper} & \textbf{Venue} & \textbf{Year} & \textbf{Key Contribution} \\
\midrule
\endhead
Kolle et al., ``Evaluating Mutation Techniques in GA-Based QCS'' \cite{kolle2025mutation} & GECCO & 2025 & Delete+swap mutations outperform others for 4--6 qubit circuits \\
\addlinespace
Sunkel, Altmann, Kolle et al., ``Hybrid EA Circuit Construction \& Optimization'' \cite{sunkel2025hybrid} & GECCO & 2025 & Hybrid EA+COBYLA achieves 80\% depth reduction at 0.98 fidelity \\
\addlinespace
``GA4QCO: GA for Quantum Circuit Optimization'' \cite{ga4qco2023} & arXiv & 2023 & GA framework for circuit optimization with Qiskit \\
\addlinespace
``GASP: GA for State Preparation'' \cite{gasp2023} & Nat.\ Sci.\ Rep. & 2023 & GA for quantum state preparation circuits \\
\addlinespace
Tandeitnik \& Guerreiro, ``Evolving Quantum Circuits'' \cite{tandeitnik2022} & arXiv & 2022 & Island-model GA for circuit decomposition \\
\addlinespace
Bhandari et al., ``EA for Boolean Gates, CA, Entanglement'' \cite{bhandari2024} & arXiv & 2024 & Mutation rate balancing for 5-qubit entangling circuits \\
\bottomrule
\end{longtable}

% --- 2.2 ---
\subsection{Quantum Architecture Search (QAS)}

\begin{longtable}{p{5.5cm} p{3cm} p{1cm} p{4.5cm}}
\toprule
\textbf{Paper} & \textbf{Venue} & \textbf{Year} & \textbf{Key Contribution} \\
\midrule
\endhead
``AQEA-QAS'' \cite{aqea2025} & MDPI Entropy & 2025 & Adaptive quantum evolutionary algorithm for QNN design \\
\addlinespace
``Noise-Aware QAS Based on NSGA-II'' \cite{noiseaware2026} & arXiv & 2026 & Multi-objective noise-aware architecture search \\
\addlinespace
``Hierarchical QAS'' \cite{hierarchical2023} & npj Quantum Inf. & 2023 & Hierarchical representations for modular search \\
\addlinespace
``Balanced QNAS'' \cite{bqnas2024} & Neurocomputing & 2024 & One-shot NAS with quantum parallelism \\
\bottomrule
\end{longtable}

% --- 2.3 ---
\subsection{QAOA + Evolutionary Methods}

\begin{longtable}{p{5.5cm} p{3cm} p{1cm} p{4.5cm}}
\toprule
\textbf{Paper} & \textbf{Venue} & \textbf{Year} & \textbf{Key Contribution} \\
\midrule
\endhead
``GA as Classical Optimizer for QAOA'' \cite{gaqaoa2023} & Appl.\ Soft Comput. & 2023 & GA outperforms gradient-free optimizers for QAOA parameters \\
\addlinespace
``QAOA Exponential Time for Linear Functions'' \cite{qaoaexp2025} & GECCO & 2025 & Fundamental complexity limitations of QAOA \\
\addlinespace
``GA-Based QAOA for Power Networks'' \cite{gaqaoapower2024} & Springer & 2024 & Domain-specific QAOA+GA evaluation \\
\bottomrule
\end{longtable}

% --- 2.4 ---
\subsection{VQE + Evolutionary Optimization}

\begin{longtable}{p{5.5cm} p{3cm} p{1cm} p{4.5cm}}
\toprule
\textbf{Paper} & \textbf{Venue} & \textbf{Year} & \textbf{Key Contribution} \\
\midrule
\endhead
``PSO for VQE (GAQPSO)'' \cite{gaqpso2024} & Phys.\ Chem.\ Chem.\ Phys. & 2024 & Gradient-free PSO with noise resilience for VQE \\
\addlinespace
Gustafson et al., ``Surrogate Optimization of VQC'' \cite{gustafson2025} & PNAS & 2025 & Classical surrogates to accelerate VQE convergence \\
\bottomrule
\end{longtable}

% --- 2.5 ---
\subsection{Multi-Objective Quantum Optimization}

\begin{longtable}{p{5.5cm} p{3cm} p{1cm} p{4.5cm}}
\toprule
\textbf{Paper} & \textbf{Venue} & \textbf{Year} & \textbf{Key Contribution} \\
\midrule
\endhead
``Quantum Approximate Multi-Objective Optimization'' \cite{qamoo2025} & Nat.\ Comp.\ Sci. & 2025 & QAOA for Pareto-optimal multi-objective solutions \\
\addlinespace
``MEAS-PQC'' \cite{measpqc2023} & MDPI Entropy & 2023 & Multi-objective EA for parameterized circuit architecture \\
\addlinespace
Zorn et al., ``Quality Diversity for VQC Optimization'' \cite{zorn2025} & Various & 2025 & CMA-MAE for quality-diversity in circuit design \\
\bottomrule
\end{longtable}

% --- 2.6 ---
\subsection{Quantum Error Correction + EA}

\begin{longtable}{p{5.5cm} p{3cm} p{1cm} p{4.5cm}}
\toprule
\textbf{Paper} & \textbf{Venue} & \textbf{Year} & \textbf{Key Contribution} \\
\midrule
\endhead
``Engineering QEC Codes with EA'' \cite{qecea2025} & IEEE TQE & 2025 & EA search for optimal stabilizer codes ($n \le 20$ qubits) \\
\addlinespace
``T-Count Optimizing GA'' \cite{tcount2024} & arXiv & 2024 & Up to 79\% T-depth reduction via GA \\
\bottomrule
\end{longtable}

% --- 2.7 ---
\subsection{ML-Based Circuit Design (Non-EA)}

\begin{longtable}{p{5.5cm} p{3cm} p{1cm} p{4.5cm}}
\toprule
\textbf{Paper} & \textbf{Venue} & \textbf{Year} & \textbf{Key Contribution} \\
\midrule
\endhead
``FlowQ-Net'' (GFlowNets) \cite{flowqnet2025} & arXiv & 2025 & 10--30$\times$ more compact circuits vs baselines \\
\addlinespace
``AlphaTensor-Quantum'' (DeepMind) \cite{alphatensor2025} & Nat.\ Mach.\ Intell. & 2025 & RL for efficient non-Clifford gate decomposition \\
\addlinespace
``Q-Fusion'' (Diffusion Model) \cite{qfusion2025} & Penn State & 2025 & Graph-based diffusion for circuit generation \\
\addlinespace
Rouillard et al., ``Automated QA Design with DSL'' \cite{rouillard2025} & arXiv & 2025 & DSL + evolutionary search rediscovers QFT, D-J, Grover \\
\bottomrule
\end{longtable}

% --- 2.8 ---
\subsection{Surveys}

\begin{longtable}{p{7cm} p{4.5cm} p{1cm}}
\toprule
\textbf{Paper} & \textbf{Venue} & \textbf{Year} \\
\midrule
\endhead
``Comprehensive Review of QCO'' \cite{qcoreview2024} & MDPI Quantum Reports / arXiv & 2024 \\
\addlinespace
``QC Synthesis \& Compilation Overview'' \cite{qcsynthesis2024} & arXiv & 2024 \\
\addlinespace
``AI for Quantum Computing'' \cite{aiqc2025} & Nature Communications & 2025 \\
\addlinespace
``Review of Procedures to Evolve Quantum Algorithms'' \cite{evolveqa2009} & GP \& Evolvable Machines & 2009 \\
\bottomrule
\end{longtable}

% ==============================================================================
\section{Standard Benchmarks in the Field}
\label{sec:benchmarks}

\begin{longtable}{p{4cm} p{5cm} p{5cm}}
\toprule
\textbf{Benchmark} & \textbf{Description} & \textbf{Typical Use} \\
\midrule
\endhead
QFT & Quantum Fourier Transform & Circuit synthesis/compilation \\
Grover's Search & Unstructured database search & EA-based circuit discovery \\
MaxCut (via QAOA) & Combinatorial optimization on graphs & QAOA parameter optimization \\
State Preparation & Prepare specific target states & GA fitness via fidelity \\
Toffoli/Fredkin Gates & Multi-qubit gate decomposition & Gate-level optimization \\
QASMBench & Suite of 1066 test circuits & Cross-framework benchmarking \\
Molecular Ground State & H$_2$, LiH, BeH$_2$ molecules & VQE ansatz optimization \\
Random Unitary Compilation & Arbitrary unitary matrices & Compiler optimization \\
Stabilizer Codes & 5-qubit, Steane, Shor codes & QEC code discovery \\
\bottomrule
\end{longtable}

\paragraph{Key metrics.} State fidelity, circuit depth, gate count (total and T-count), approximation ratio, quantum volume, Hellinger fidelity, process fidelity, and hypervolume indicator (multi-objective).

% ==============================================================================
\section{Open Problems}
\label{sec:open}

The following open problems were identified across the surveyed literature:

\begin{enumerate}
    \item \textbf{Scalability.} Most EA approaches are tested on 4--8 qubits. Classical statevector simulation becomes infeasible beyond ${\sim}30$ qubits.
    \item \textbf{Continuous parameters.} Handling rotation gate parameters ($R_x$, $R_y$, $R_z$) within discrete evolutionary search remains an open challenge.
    \item \textbf{Noise-aware optimization.} Most EAs assume ideal execution; NISQ relevance requires noise models.
    \item \textbf{Standardized benchmarks.} No universally accepted benchmark suite exists for EA-based quantum circuit synthesis.
    \item \textbf{Search space representation.} The optimal circuit encoding for evolutionary operators is still debated.
    \item \textbf{Hardware--software co-design.} Joint optimization with hardware topology constraints is underexplored.
    \item \textbf{Multi-objective trade-offs.} Richer objective spaces beyond fidelity and depth are largely unexplored.
    \item \textbf{Transfer and generalization.} Evolved circuits rarely generalize across problem instances or hardware.
    \item \textbf{Verification at scale.} Verifying evolved circuits becomes intractable for large qubit counts.
    \item \textbf{Adaptive operators.} Self-adaptive mutation and crossover rates specifically for quantum circuit EAs are understudied.
\end{enumerate}

% ==============================================================================
\section{Gap Analysis: This Project vs.\ the Literature}
\label{sec:gap}

\subsection{Project Summary}

This project implements a comparative study of six optimization methods (EA, Random Search, Gradient-Based, REINFORCE, NSGA-II, MAP-Elites) for evolving quantum circuits across six benchmark problems (Grover, Flip, Inverse, Fourier, Deutsch--Jozsa, Bernstein--Vazirani) using:
\begin{itemize}
    \item Pure numpy statevector simulation (${\sim}\mu$s per evaluation),
    \item DEAP evolutionary algorithm framework,
    \item 3-qubit circuits with 5 discrete gate types (I, T, H, CNOT$_\downarrow$, CNOT$_\uparrow$),
    \item Single-objective and multi-objective fitness (fidelity, depth, gate count),
    \item Quality-diversity optimization via MAP-Elites,
    \item Fitness caching keyed by circuit bytes.
\end{itemize}

\subsection{Strengths vs.\ Literature}

\begin{longtable}{p{4.5cm} p{9.5cm}}
\toprule
\textbf{Strength} & \textbf{How It Compares} \\
\midrule
\endhead
Simulation speed (${\sim}\mu$s/eval) & 100--1000$\times$ faster than Qiskit Aer used in GA4QCO, GASP, etc. \\
\addlinespace
Fitness caching & Rarely seen in published frameworks \\
\addlinespace
DEAP integration & Mature, well-tested; most papers use custom implementations \\
\addlinespace
Six-method comparison & EA, Random, Gradient, DL, NSGA-II, MAP-Elites compared head-to-head \\
\addlinespace
Quality-diversity & MAP-Elites for algorithmic quantum circuits is nearly unexplored \\
\addlinespace
Reproducibility & Deterministic seeds, 10 trials per condition, 360 total runs \\
\addlinespace
No runtime Qiskit dependency & Eliminates version churn, import overhead \\
\bottomrule
\end{longtable}

\subsection{Gaps vs.\ Literature}

\begin{longtable}{p{4cm} p{1.5cm} p{8.5cm}}
\toprule
\textbf{Gap} & \textbf{Severity} & \textbf{What the Literature Does} \\
\midrule
\endhead
Fixed discrete gate set only & HIGH & GA4QCO, GASP, GECCO~2025 all support parameterized rotation gates ($R_x$, $R_y$, $R_z$) \\
\addlinespace
No noise model & HIGH & Noise-Aware QAS (2026), GA-QAOA on real hardware show noise-aware fitness is critical for NISQ \\
\addlinespace
Only 3-qubit experiments & MEDIUM & Most papers test on 4--6+ qubits \\
\addlinespace
Limited gate set & MEDIUM & No $R_x$, $R_y$, $R_z$, CZ, Toffoli gates; limits circuit expressiveness \\
\addlinespace
No adaptive operators & MEDIUM & GECCO~2025 shows benefits of adaptive mutation rates \\
\addlinespace
No hardware topology constraints & LOW & Compilation-aware synthesis is an emerging direction \\
\addlinespace
No transfer/generalization study & LOW & DSL paper \cite{rouillard2025} shows generalizable algorithm learning \\
\bottomrule
\end{longtable}

\subsection{Key Positioning}

The project occupies a practical niche: \textbf{a lightweight, fast, dependency-minimal framework for evolving small quantum circuits with multi-objective and quality-diversity optimization}. Most comparable to GA4QCO and GASP, but differentiated by simulation speed, DEAP integration, and the inclusion of NSGA-II and MAP-Elites. The main remaining gaps are the lack of continuous parameters, noise modelling, and larger qubit counts.

% ==============================================================================
\section{Emerging Topics and Niche Opportunities}
\label{sec:emerging}

\subsection{Quality-Diversity for Quantum Circuits}
\textbf{Status: Very under-explored.}
Only one paper (Zorn et al.\ \cite{zorn2025}, CMA-MAE for VQC optimization on MaxCut/MVC/MIS/MaxClique) applies quality-diversity to quantum circuits. No work applies MAP-Elites to algorithmic quantum circuit discovery (QFT, Grover, etc.).
Our project addresses this gap directly, using MAP-Elites with (depth, entanglement density) as behavioural descriptors to reveal fundamentally different circuit families for the same problem.

\subsection{Noise-Aware Evolutionary Circuit Synthesis}
\textbf{Status: Nascent.}
Only one direct paper \cite{noiseaware2026} applies NSGA-II with noise models, using Qiskit's noisy simulation (slow). No fast numpy-based noisy simulation for EA fitness evaluation exists. Opportunity: implement depolarizing/amplitude damping noise in numpy and show that noise-aware evolution produces more hardware-robust circuits.

\subsection{Surrogate-Assisted Evolutionary Quantum Circuit Optimization}
\textbf{Status: Very new.}
Gustafson et al.\ \cite{gustafson2025} pioneered surrogate optimization for VQC but used gradient-based methods, not EA. No surrogate-assisted EA specifically for quantum circuit structure optimization exists.

\subsection{Transfer Learning for Evolved Quantum Circuits}
\textbf{Status: Nearly empty.}
Rouillard et al.\ \cite{rouillard2025} showed DSL-based circuits learned on 5 qubits generalize to larger instances, but no EA-based transfer learning work exists. Open questions: can circuits evolved for 3-qubit problems seed evolution for 4--5 qubit problems? Can circuits evolved for Grover transfer to Bernstein--Vazirani?

\subsection{Grammatical Evolution for Quantum Circuits}
\textbf{Status: Sparse but promising.}
A 2025 paper on grammatical evolution for Grover achieved 97.9\% fidelity on IBM hardware vs 44.2\% for standard Grover, with up to 93.3\% depth reduction. Formalizing gate constraints as a BNF grammar could eliminate wasted evaluations on invalid circuits.

\subsection{Co-Evolution for Quantum Circuits}
\textbf{Status: Empty.}
No true co-evolutionary approach to quantum circuit design exists. Co-evolving circuit structure and test cases, or co-evolving complementary circuit sub-modules, is completely unexplored.

\subsection{Evolutionary Quantum Error Mitigation}
\textbf{Status: Nascent.}
One paper (IEEE 2021) uses GA for measurement error mitigation. No EA-based search for optimal error mitigation protocols (zero-noise extrapolation parameters, probabilistic error cancellation strategies) exists.

\subsection{Summary of Research Whitespace}

\begin{table}[h]
\centering
\begin{tabular}{lcc}
\toprule
\textbf{Topic} & \textbf{Papers Found} & \textbf{Opportunity Level} \\
\midrule
Quality-Diversity / MAP-Elites for QCS & 1 & Very High \\
Transfer Learning for Evolved Circuits & 0 & High \\
Co-Evolution for Quantum Circuits & 0 & High \\
Surrogate-Assisted EA for QCS & 0 (direct) & High \\
Noise-Aware EA for QCS & 1 & High \\
Grammatical Evolution for QCS & 1 (tangential) & Medium--High \\
Adaptive Mutation for QCS & 1 & Medium--High \\
Evolutionary Quantum Error Mitigation & 1 (tangential) & Medium \\
\bottomrule
\end{tabular}
\caption{Research whitespace assessment as of February 2026.}
\label{tab:whitespace}
\end{table}

% ==============================================================================
\section{Publishable Contributions}
\label{sec:contributions}

Based on the literature review and gap analysis, five concrete publication-ready research directions are identified, ranked by feasibility and novelty.

\subsection{Rec.\ 1: Multi-Objective NSGA-II for Quantum Circuit Synthesis}

\textbf{Novelty: High. Status: Implemented.}
No published work combines DEAP's built-in NSGA-II with fast numpy statevector simulation for multi-objective quantum circuit optimization. The closest work (MEAS-PQC \cite{measpqc2023}; Noise-Aware QAS \cite{noiseaware2026}) uses custom implementations or Qiskit.
Our project implements this with three objectives (maximize fidelity, minimize active depth, minimize gate count) using \texttt{selNSGA2} and \texttt{eaMuPlusLambda}.

Target venues: GECCO~2026, CEC~2026, Quantum Science and Technology.

\subsection{Rec.\ 2: Adaptive Mutation Strategies for Quantum Circuit Evolution}

\textbf{Novelty: Medium--High.}
Kolle et al.\ \cite{kolle2025mutation} evaluated fixed mutation strategies but did not study self-adaptive mutation rates or quantum-aware mutation operators.
Possible extensions: gate substitution preserving unitarity, subcircuit inversion, controlled-gate promotion, 1/5 success rule adaptation, multi-armed bandit operator selection.

Target venues: GECCO~2026, IEEE Transactions on Evolutionary Computation.

\subsection{Rec.\ 3: MAP-Elites Quality-Diversity for Circuit Repertoires}

\textbf{Novelty: Very High. Status: Implemented.}
Nobody has applied MAP-Elites to discover diverse repertoires of quantum circuits for algorithmic problems (Grover, QFT, etc.). Instead of finding one best circuit, MAP-Elites finds a map of diverse high-quality circuits indexed by structural features.
Our project implements this with a 2D archive indexed by (active depth, entanglement density), achieving 79--87\% coverage across problems.

Target venues: GECCO~2026, Artificial Life, Evolutionary Computation.

\subsection{Rec.\ 4: Noise-Aware Evolutionary Synthesis with Surrogates}

\textbf{Novelty: Medium. Practical impact: High.}
Combine fast numpy noise models (depolarizing channel via Kraus operators or density matrix simulation) with surrogate-assisted evaluation (random forest or GP regression on circuit features $\to$ noisy fitness).

Target venues: Quantum Science and Technology, Physical Review A, IEEE TQE.

\subsection{Rec.\ 5: Parameterized Gate Evolution with Hybrid EA+Local Search}

\textbf{Novelty: Medium.}
Extend the gate set with $R_x(\theta)$, $R_y(\theta)$, $R_z(\theta)$ and continuous angle parameters. EA evolves gate topology; local optimizer (L-BFGS-B or COBYLA) tunes parameters.

Target venues: GECCO~2026, J.~Chemical Theory and Computation.

\subsection{Rec.\ 6: Grammatical Evolution with Hardware Constraints}

\textbf{Novelty: Medium--High.}
Formalize existing gate preprocessing rules (CNOT adjacency, H/T cancellation) as a BNF grammar. Use grammatical evolution instead of flat integer encoding. A 2025 GE paper achieved 97.9\% fidelity on real IBM hardware for 3-qubit Grover; expanding to our 6-problem benchmark suite would be a clear extension.

Target venues: GECCO~2026, Evolutionary Computation.

\subsection{Summary}

\begin{table}[h]
\centering
\begin{tabular}{clcccc}
\toprule
\textbf{\#} & \textbf{Direction} & \textbf{Novelty} & \textbf{Feasibility} & \textbf{Effort} & \textbf{Status} \\
\midrule
1 & Multi-Objective NSGA-II & High & High & 2--4 wk & Done \\
2 & Adaptive Mutation & Med--High & High & 3--4 wk & Future \\
3 & MAP-Elites QD & Very High & Medium & 3--5 wk & Done \\
4 & Noise-Aware + Surrogate & Medium & Medium & 4--6 wk & Future \\
5 & Parameterized Gates Hybrid & Medium & Medium & 4--6 wk & Future \\
6 & Grammatical Evolution & Med--High & Medium & 3--4 wk & Future \\
\bottomrule
\end{tabular}
\caption{Summary of publication-ready research directions.}
\label{tab:recommendations}
\end{table}

The recommended combined paper---\emph{``Multi-Objective and Quality-Diversity Optimization of Quantum Circuits via Statevector Simulation''}---combining Recommendations~1 and~3, has been implemented and is presented in the accompanying research paper.

% ==============================================================================
\section{Standout Recent Results}
\label{sec:standout}

Several additional results merit attention:

\paragraph{Grammatical Evolution for Grover (2025).}
Evolved Grover circuits for all 8 basis states of a 3-qubit system on IBM \texttt{ibm\_brisbane} achieved 97.9\% fidelity vs 44.2--47.6\% for standard Grover, with up to 93.3\% depth reduction and 92.7\% gate count reduction.

\paragraph{EXAQC (RIT).}
Neuroevolution-style search that simultaneously optimizes gate types, qubit connectivity, parameterization, and circuit depth. Achieves $>$90\% accuracy on Iris, Wine, Seeds, and Breast Cancer benchmarks. Supports both Qiskit and PennyLane.

\paragraph{QuantumNAS (Wang et al., HPCA 2022).}
SuperCircuit approach that decouples training from search. Noise-adaptive evolutionary co-search for (circuit, qubit mapping) pairs, tested on 14 quantum computers across 12 benchmarks.

\paragraph{Evolutionary BP+OSD Decoding for QEC (Dec 2025).}
Differential evolution optimizes belief propagation weights, achieving comparable performance with 5 BP iterations vs 32 (standard) or 150 (BP with memory).

\paragraph{RBF Surrogate for 127-Qubit QAOA (2025).}
Radial basis function interpolation as an adaptive, hyperparameter-free surrogate successfully optimized 127-qubit QAOA circuits on IBM hardware.

% ==============================================================================
\section{Key Active Research Groups}
\label{sec:groups}

\paragraph{LMU Munich (Linnhoff-Popien group).}
The most prolific group in the field: Kolle (mutation strategies, GECCO~2025), Altmann (hybrid EA, GECCO~2025), Zorn \& Stein (quality-diversity, ICAPS~2025), Sunkel (circuit construction, GECCO~2025), Gabor (QNEAT, GECCO~2023).

\paragraph{MIT HAN Lab (Wang).} QuantumNAS---noise-adaptive co-search.

\paragraph{University of KwaZulu-Natal (Rouillard, Petruccione).} DSL-based algorithm design.

\paragraph{RIT (Kar, Krutz, Desell).} EXAQC neuroevolution.

\paragraph{OsloMet (Bhandari, Nichele, Lind).} EA for entanglement.

\paragraph{University of Melbourne (Creevey, Hill, Hollenberg).} GASP.

% ==============================================================================
\section{Software Toolkits}
\label{sec:tools}

\begin{table}[h]
\centering
\begin{tabular}{p{3.5cm} p{7cm} p{3.5cm}}
\toprule
\textbf{Tool} & \textbf{Description} & \textbf{Reference} \\
\midrule
EVOVAQ & Python toolbox for evolutionary VQC training (Qiskit) & GitHub \\
EXAQC & Neuroevolution for quantum circuits (Qiskit + PennyLane) & RIT \\
AlphaTensor-Quantum & DeepMind RL for T-count optimization & \cite{alphatensor2025} \\
MQT Bench & ${\sim}$70,000 benchmark circuits, 2--130 qubits & GitHub \\
Benchpress & 1000+ tests for circuit compilation (up to 930 qubits) & Nat.\ Comp.\ Sci.\ 2025 \\
RevLib & Reversible function/circuit benchmarks & revlib.org \\
\bottomrule
\end{tabular}
\caption{Key software toolkits for quantum circuit optimization.}
\label{tab:tools}
\end{table}

% ==============================================================================
\section{Target Venues}
\label{sec:venues}

\begin{itemize}
    \item \textbf{GECCO 2026} (ACM Genetic and Evolutionary Computation Conference)---the premier venue; GECCO~2025 had multiple quantum circuit EA papers and dedicated quantum optimization workshops.
    \item \textbf{CEC 2026} (IEEE Congress on Evolutionary Computation).
    \item \textbf{IEEE QCE 2026} (IEEE International Conference on Quantum Computing and Engineering).
    \item \textbf{QIP 2026} (Quantum Information Processing).
    \item \textbf{Quantum Science and Technology} (IOP journal).
    \item \textbf{Evolutionary Computation} (MIT Press journal).
    \item \textbf{IEEE Transactions on Quantum Engineering}.
\end{itemize}

% ==============================================================================
\section*{Acknowledgements}

Literature search and synthesis conducted in February 2026.

% ==============================================================================
\begin{thebibliography}{35}

\bibitem{kolle2025mutation}
M.~Kolle et al.,
``Evaluating mutation techniques in genetic algorithm-based quantum circuit synthesis,''
in \emph{Proc.\ GECCO}, 2025.
\newblock \href{https://arxiv.org/abs/2504.06413}{arXiv:2504.06413}.

\bibitem{sunkel2025hybrid}
L.~Sunkel, P.~Altmann, M.~Kolle et al.,
``Quantum circuit construction and optimization through hybrid evolutionary algorithms,''
in \emph{Proc.\ GECCO}, 2025.
\newblock \href{https://arxiv.org/abs/2504.17561}{arXiv:2504.17561}.

\bibitem{ga4qco2023}
``GA4QCO: Genetic algorithm for quantum circuit optimization,''
2023.
\newblock \href{https://arxiv.org/abs/2302.01303}{arXiv:2302.01303}.

\bibitem{gasp2023}
``GASP: Genetic algorithms for state preparation on quantum computers,''
\emph{Nature Scientific Reports}, 2023.
\newblock \href{https://www.nature.com/articles/s41598-023-37767-w}{doi:10.1038/s41598-023-37767-w}.

\bibitem{tandeitnik2022}
D.~Tandeitnik and T.~Guerreiro,
``Evolving quantum circuits,''
2022.
\newblock \href{https://arxiv.org/abs/2210.05058}{arXiv:2210.05058}.

\bibitem{bhandari2024}
A.~Bhandari, S.~Nichele, A.~Denysov, and P.~Lind,
``Evolutionary algorithm for quantum circuits: Boolean gates, cellular automata, and entanglement,''
2024.
\newblock \href{https://arxiv.org/abs/2408.00448}{arXiv:2408.00448}.

\bibitem{aqea2025}
``AQEA-QAS: Adaptive quantum evolutionary algorithm for quantum architecture search,''
\emph{MDPI Entropy}, vol.~27, no.~7, 2025.

\bibitem{noiseaware2026}
``Noise-aware quantum architecture search based on NSGA-II algorithm,''
2026.
\newblock \href{https://arxiv.org/abs/2601.10965}{arXiv:2601.10965}.

\bibitem{hierarchical2023}
``Hierarchical quantum circuit representations for neural architecture search,''
\emph{npj Quantum Information}, 2023.
\newblock \href{https://www.nature.com/articles/s41534-023-00747-z}{doi:10.1038/s41534-023-00747-z}.

\bibitem{bqnas2024}
``Balanced quantum neural architecture search,''
\emph{Neurocomputing}, 2024.

\bibitem{gaqaoa2023}
``Genetic algorithm as a classical optimizer for QAOA,''
\emph{Applied Soft Computing}, 2023.

\bibitem{qaoaexp2025}
``The QAOA can require exponential time to optimize linear functions,''
in \emph{Proc.\ GECCO}, 2025.

\bibitem{gaqaoapower2024}
``GA-based QAOA for power networks,''
Springer, 2024.

\bibitem{gaqpso2024}
``Particle swarm optimization for VQE (GAQPSO),''
\emph{Physical Chemistry Chemical Physics}, 2024.

\bibitem{gustafson2025}
E.~Gustafson et al.,
``Surrogate optimization of variational quantum circuits,''
\emph{PNAS}, 2025.
\newblock \href{https://www.pnas.org/doi/10.1073/pnas.2408530122}{doi:10.1073/pnas.2408530122}.

\bibitem{qamoo2025}
``Quantum approximate multi-objective optimization,''
\emph{Nature Computational Science}, 2025.
\newblock \href{https://www.nature.com/articles/s43588-025-00873-y}{doi:10.1038/s43588-025-00873-y}.

\bibitem{measpqc2023}
``MEAS-PQC: Multi-objective evolutionary algorithm search for parameterized quantum circuit architecture,''
\emph{MDPI Entropy}, vol.~25, no.~1, 2023.

\bibitem{zorn2025}
M.~Zorn, J.~Stein, M.~Kolle et al.,
``Quality diversity for variational quantum circuit optimization,''
2025.

\bibitem{qecea2025}
``Engineering quantum error correction codes using evolutionary algorithms,''
\emph{IEEE Trans.\ Quantum Engineering}, 2025.
\newblock \href{https://arxiv.org/abs/2409.13017}{arXiv:2409.13017}.

\bibitem{tcount2024}
``T-count optimizing genetic algorithm for quantum state preparation,''
2024.
\newblock \href{https://arxiv.org/abs/2406.04004}{arXiv:2406.04004}.

\bibitem{flowqnet2025}
``FlowQ-Net: A generative framework for automated quantum circuit design,''
2025.
\newblock \href{https://arxiv.org/abs/2510.26688}{arXiv:2510.26688}.

\bibitem{alphatensor2025}
``AlphaTensor-Quantum,''
\emph{Nature Machine Intelligence}, 2025.
\newblock \href{https://www.nature.com/articles/s42256-025-01001-1}{doi:10.1038/s42256-025-01001-1}.

\bibitem{qfusion2025}
``Q-Fusion: Diffusion-based quantum circuit generation,''
Penn State, 2025.

\bibitem{rouillard2025}
R.~Rouillard, B.~Lourens, and F.~Petruccione,
``Automated quantum algorithm design via domain-specific language,''
2025.
\newblock \href{https://arxiv.org/abs/2503.08449}{arXiv:2503.08449}.

\bibitem{qcoreview2024}
``Comprehensive review of quantum circuit optimization,''
\emph{MDPI Quantum Reports}, 2024.
\newblock \href{https://arxiv.org/abs/2408.08941}{arXiv:2408.08941}.

\bibitem{qcsynthesis2024}
``Quantum circuit synthesis and compilation: Overview and prospects,''
2024.
\newblock \href{https://arxiv.org/abs/2407.00736}{arXiv:2407.00736}.

\bibitem{aiqc2025}
``Artificial intelligence for quantum computing,''
\emph{Nature Communications}, 2025.
\newblock \href{https://www.nature.com/articles/s41467-025-65836-3}{doi:10.1038/s41467-025-65836-3}.

\bibitem{evolveqa2009}
``A review of procedures to evolve quantum algorithms,''
\emph{Genetic Programming and Evolvable Machines}, 2009.

\end{thebibliography}

\end{document}
